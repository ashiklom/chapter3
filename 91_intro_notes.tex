%%%%%%%%%%%%%%%%%%%%%%%%%%%%%%%%%%%%%%%%%%%%%%%%%%%%%%%%%%%%

using unmanned aerial systems (CITE), airplanes (CITE), and satellites.

In addition to their direct utility to trait ecology, analyses of leaf spectra are also essential for solidifying relationships between leaf spectra and traits that can be us

For some traits

In cases where traits have been evaluated in experimental contexts, 

Spectral measurements are capable of providing fast and non-destructive assessments of plant traits, which can be used

Remote sensing is emerging as a powerful method for mapping and monitoring plant traits over large areas and through time \cite{schneider2017_mapping,schimel2013_observing,schimel2015_observing,jetz2016_diversity}.
Numerous studies have demonstrated the success of mapping traits from unmanned aerial systems (CITE), airborne platforms \cite{schneider2017_mapping,asner2017_scale,singh2015_istraits}, and satellites (CITE: Houborg, others).
At the scale of leaves and individual plant canopies, spectral measurements with portable spectroradiometers provide fast and non-destructive assessments of foliar traits that facilitate upscaling of measurements and tracking of physiological responses (CITE: Serbin, Martin, Penuelas, others).

%%%%%%%%%%%%%%%%%%%%%%%%%%%%%%%%%%%%%%%%%%%%%%%%%%%%%%%%%%%

More on intra-specific trait variability:

  Variability in litter leachate chemistry by site and phenophase \cite{hudson_2018_american}.

  Variability in traits within a canopy, using leaf spectra \cite{gara_2018_impact}

Applications of leaf spectra (and general leaf spectra puff piece) \cite{cavenderbares_2017_harnessing}.

    Importance of field spectral measurements to improve interpretation of remote sensing spectra

Imaging spectroscopy related to traits and phylogeny \cite{mcmanus_2016_phylogenetic}

IS to study plant secondary metabolites \cite{couture_2016_spectroscopic}

IS (airborne) to map foliar nitrogen. In particular, used PROSPECT RTM inversion \cite{wang_2018_mapping}.

  PROSPECT recalibration and inversion to estimate proteins \cite{wang_2015_applicability}

  RTM inversion to estimate N from other traits \cite{wang_2015_leaf}

Spectroscopy can be used to monitor plant stress

    Maize plant response to drought \cite{sun_2018_reflectance}

Detection of invasive species \cite{ewald_2018_analyzing}

%%%%%%%%%%%%%%%%%%%%%%%%%%%%%%%%%%%%%%%%%%%%%%%%%%%%%%


As I discussed previously, plant functional traits are useful for understanding and predictive modeling of plant function.
Existing global trait databases such as TRY are useful for studying large-scale patterns of plant functional variability.
However, these databases are spatially incomplete, and, more importantly, do not typically sample through time, precluding their use in analyses of plant functional responses to a changing environment.
In both regards, remote sensing is a potentially useful complement to traditional trait measurements due to its spatial completeness and ability to sample through time.
Radiative transfer models (RTMs) provide a useful, physically-based link between plant traits and the optical features measured by remote sensing techniques.



In Chapter 1, I discussed the importance of plant functional traits to dynamic vegetation modeling and ecological research more generally.
Using a global database of plant functional traits (TRY), that work was able to explore the scale dependence of large-scale patterns of trait correlation.
Then, in Chapter 2, I described the limitations of static trait databases to answering questions related to global change ecology and presented remote sensing as an essential comlpetement for mapping traits through time and with greater spatial completeness.
Therein, I also presented a methodology for estimating leaf traits from leaf spectra.
In this Chapter, I further develop this methodology and apply it to a large collection of leaf spectral data.
This Chapter is an application of this method across a large collection of leaf spectral data

However, there are fundamental limitations to the sorts of ecological questions that can be answered using TRY and similar trait databases.

For one, sampling within TRY is not representative geographically or phylogenetically, as evidenced by the larger uncertainties in mean trait estimates for certain plant functional types (e.g.,~those from high-latitude biomes).

Plant functional traits are critical to understanding plant function.
Traits often serve as model parameters (CITE MODELING PAPERS), so better estimates of plant traits can lead to better constraints on carbon cycling \cite{Dietze_2014,Dietze_2013_modeldata,LeBauer_2013_pecan}.
For example, more trait measurements are key to accurate parameterization and representation of photosynthesis \cite{rogers2017_roadmap,rogers2017_vcmax}
More flexible representations of plant traits in models have been used in models \cite{vanbodegom2012_beyond,verheijen2015_inclusion,vanbodegom2014_fully,verheijen2015_variation}
Beyond model parameters, trait analyses can influence model structure and design by elucidating drivers of plant function in novel ways.
For instance, climatic drivers of leaf size \cite{Wright_2017_leafsize}.
Effects of traits on community dynamics, including competition \cite{kunstler2016_competition,rosado2017_csr}.
As a result, traits have been identified as key essential biodiversity variables essential to monitoring the progress of global conservation initiatives \cite{tittensor2014_targets,pereira2013_ebv,geijzendorffer2015_ebv}.

Syntheses of global trait databases (e.g. TRY, Kattge et al.~2011; FRED, Iversen et al. 2017) \nocite{kattge2009_try,iversen2017_fred} have been useful for lots of studies.
However, these databases are incomplete.
Large spatial gaps in sampling, particularly biomes critical to the global climate system like boreal and tropical forests \cite{jetz2016_diversity}.
Moreover, even where these measurements are complete, they are incapable of providing data through time, which is critical to understanding how ecosystems respond directly to a changing world (CITE).
Remote sensing is emerging as a powerful method for mapping and monitoring plant traits over large areas and through time \cite{schneider2017_mapping,schimel2013_observing,schimel2015_observing,jetz2016_diversity}.
Numerous studies have demonstrated the success of mapping traits from unmanned aerial systems (CITE), airborne platforms \cite{schneider2017_mapping,asner2017_scale,singh2015_istraits}, and even satellites (CITE: Houborg, others).
At the scale of leaves and individual plant canopies, spectral measurements with portable spectroradiometers provide fast and non-destructive assessments of foliar traits  that facilitate upscaling of measurements and tracking of physiological responses (CITE: Serbin, Martin, Penuelas, others).

More on intra-specific trait variability:

  Variability in litter leachate chemistry by site and phenophase \cite{hudson_2018_american}.

  Variability in traits within a canopy, using leaf spectra \cite{gara_2018_impact}

Applications of leaf spectra (and general leaf spectra puff piece) \cite{cavenderbares_2017_harnessing}.

    Importance of field spectral measurements to improve interpretation of remote sensing spectra

Imaging spectroscopy related to traits and phylogeny \cite{mcmanus_2016_phylogenetic}

IS to study plant secondary metabolites \cite{couture_2016_spectroscopic}

IS (airborne) to map foliar nitrogen. In particular, used PROSPECT RTM inversion \cite{wang_2018_mapping}.

  PROSPECT recalibration and inversion to estimate proteins \cite{wang_2015_applicability}

  RTM inversion to estimate N from other traits \cite{wang_2015_leaf}

Spectroscopy can be used to monitor plant stress

    Maize plant response to drought \cite{sun_2018_reflectance}

Detection of invasive species \cite{ewald_2018_analyzing}

In this study, we focus on six foliar "optical" traits related to plant ecophysiology and spectral properties.
The effective number of leaf mesophyll layers (or one more than the effective number of intracellular air spaces)...(CITE).
Total leaf chlorophyll content (the sum of chlorophyll $a$ and $b$) is relevant to photosynthesis and APAR (CITE: Croft, others).
Total leaf carotenoid content (introduced in PROSPECT 5) is related to the xanthophyll cycle, a key mechanism for preventing plant photooxidiative stress (CITE).
Total leaf anthocyanin content (introduced in PROSPECT D) has a similar role in boosting plant stress tolerance, and does something related to fall phenology (CITE).
Leaf water content is related to leaf water stress (CITE).
Leaf dry matter content is related to...(CITE).

Collectively, these traits can be used to simulate leaf reflectance and transmittance using the PROSPECT leaf radiative transfer model \cite{jacquemoud1990_prospect,feret2008_prospect,feret2017_prospectd}.
PROSPECT has been used extensively for the simulation of leaf and (combined with canopy models) canopy reflectance (CITE).
PROSPECT has also been used to estimate leaf spectral characteristics through spectral inversion (CITE).
Compared to alternative approaches for estimating leaf properties from spectra, including spectral indices (CITE) and partial least squares regression (PLSR) \cite{barnes_2017_beyond} (CITE: Serbin), PROSPECT inversion offers the following advantages.
First, RTM inversion can be applied to virtually any instrument, because the spectra simulated by RTMs can then be downsampled to any spectral resolution based on the spectral response curve of the corresponding instrument.
Combined with Bayesian algorithms, this allows RTM inversion to account for the additional uncertainties associated with collecting spectra using instruments of various resolutions (Chapter 2).
Second, the ability of RTMs to forward-simulate vegetation-light interactions makes them valuable to other applications, such as modeling leaf absorbance for photosynthesis and for improving representations of energy balance in terrestrial biosphere models (CITE, Fisher 2017?).
Another compelling argument for the use RTM inversion over empirical approaches is that RTMs are designed to be generic across all species and conditions, whereas empirical approaches must be calibrated to the available data.
However, only one coefficient in the PROSPECT model---that of leaf water content---is based on the absorption spectrum of the corresponding leaf property,the remaining coefficients for pigments are empirically calibrated.
This empirical calibration poses challenges for the application of PROSPECT inversion, particularly for species dissimilar from those used in its calibration (such as arid shrubs, e.g. CITE); however, a comprehensive validation of PROSPECT over a wide range of species and measurement conditions has not previously been attempted to our knowledge.

This study addresses three questions.
First, how well can leaf optical traits be estimated from PROSPECT inversion over a wide range of species and experimental designs?
Second, how do leaf optical traits vary across a variety of environmental conditions and species?
  Through natural and experimental physiological disturbances, including insect and pathogen damage, drought stress, and fertilization treatments
  Through phenological changes
Third, how are leaf optical traits related to other leaf traits not directly estimable from PROSPECT inversion?
To address these questions, we compiled a database of leaf spectra and, where available, trait measurements collected in both experimental settings and in the field for a wide range of projects.
For each spectral observation, we applied a modified version of the Bayesian spectral inversion approach in Chapter 2 to estimate the leaf optical traits.
To validate the PROSPECT inversion, we compared the spectral inversion estimates to direct measurements of the same traits, where such measurements were available.
Where spectra and traits were collected from the same species but under a variety of conditions (e.g., experimental treatment, disturbance), we investigated how leaf optical traits varied across these conditions.
We then selected the control groups from these settings, combined them with trait data sampled randomly in the field, and partitioned their variability based on ecologically-meaningful contrasts.
Finally, we investigated the correlations between optical traits and other traits sampled directly.
